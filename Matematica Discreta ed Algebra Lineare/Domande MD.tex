\documentclass[]{article}
\usepackage[utf8]{inputenc}
\usepackage[italian]{babel}
\usepackage{amsmath}
\usepackage{amssymb}
\usepackage{amsthm}
\usepackage{cancel}
\usepackage{mathrsfs}
\usepackage{subfigure}
\usepackage{hyperref}
\usepackage{titlesec}
%\newcommand{\sectionbreak}{\clearpage}

\begin{document}
\newcommand{\campo}[1]{\mathbb{#1}}
\newcommand{\minug}{\leqslant}
\newtheorem*{lem}{Teorema}
\newtheorem*{definizione}{Definizione}
\title{Risposte Orale Breve MD}
\author{Alessandro Pagiaro}
\date{Aggiornato al 23 Giugno 2014}
\maketitle
\tableofcontents
\clearpage
\section{Domanda 2 - Somma dei primi n numeri...}
$$\sum_{i=0}^n i = \frac{(n+1)n}{2}$$Questo risultato si dimostra facilmente per via grafica. Basterà immaginare un triangolo rettangolo con i cateti di lunghezza $n$ e calcolarne l'area.\\
Per quanto riguarda la somma dei quadrati possiamo procedere in maniera analoga e ottenere che $$\sum_{i=0}^n i^2  = \frac{n(n+1)(2n+1)}{6}$$
\href{http://www.batmath.it/matematica/avista/somma_quadr/somma_quadr.htm}{Approfondisci $>>$}\\
\section{Domanda 3 - Induzione}
Gli esercizi così si risolvono trovando il minimo numero $n \in \campo{N}$ per cui vale la relazione data e la si dimostra per induzione su $n$. 
\section{Domanda 4 - Fare una dimostrazione del Principio del minimo}
\begin{lem}[Principio del minimo o del Buon Ordinamento]
Ogni sottoinsieme non vuoto di $\campo{N}$ ha un elemento minimo.
\end{lem}
Una dimostrazione dove lo si usa è quella del Teorema di Bezout. 
\section{Domanda 8 - $g\circ f$ iniettività}
Data $f:X \rightarrow Y$ e $g: Y \rightarrow Z$\\
Data $g\circ f: X \rightarrow Z$ iniettiva, è vero che:
\begin{itemize}
\item $\Rightarrow f$ iniettiva
\item $\Rightarrow g$ iniettiva
\end{itemize}
VERO!
\begin{proof}
difatti, se una delle due non fosse iniettiva potrei avere una $\bar{x}$ e una $\bar{y}$ con $\bar{x} \not= \bar{y}$ tale che $f(\bar{x})=f(\bar{y})$ per cui $g(f(\bar{x}))=g(f(\bar{y}))$ con $\bar{x} \not= \bar{y}$. Cioè avrei $g\circ f$ non iniettiva, assurdo, visto che per ipotesi la composizione è iniettiva.
\end{proof}
Si dimostra in maniera analoga che $g$ deve essere iniettiva.
\section{Domanda 9 - $g\circ f$ surgettività}
Data $f:X \rightarrow Y$ e $g: Y \rightarrow Z$\\
Data $g\circ f: X \rightarrow Z$ surgettiva, è vero che:
\begin{itemize}
\item $\Rightarrow f$ surgettiva
\item $\Rightarrow g$ surgettiva
\end{itemize}
Se $g \circ f$ è surgettiva $\Rightarrow |X| \supseteq |Z|$.\\
È FALSO che $\Rightarrow f$ surgettiva. Costruisco un esempio che mi nega l'affermazione.\\
Prendo $X = \{1,2,3,4\} = \campo{N}_4, Y = \campo{N}_5, Z = \campo{N}_4$.
Definisco ora: 
$$f(x) = x$$
$$g(x) = \bigg \{ 
\begin{array}{lr}
x & se\ x \in \{1,2,3,4\}\\
4 & se\ x = 5
\end{array}
$$
La composizione risulterà chiaramente surgettiva eppure la $f$ non è surgettiva, difatti l'elemento $5 \in Y$ non viene mai raggiunto da alcuna $x \in X$ eppure $\forall z \in Z\  \exists x | f\circ g(x)=z$.\\
È VERO che $\Rightarrow g$ surgettiva. 
\begin{proof}Se infatti non fosse così, poichè per definizione di funzione $$g \circ f = g(f(x))$$ non riuscirei a raggiungere un elemento in $Z$ poichè qualunque elemento $x$ scelgo, $g(f(x))$ non lo "raggiungerebbe" e quindi $g \circ f$ risulterebbe non surgettiva. Ma questo va contro la nostra ipotesi iniziale.\end{proof}
\section{Domanda 12 - ${n \choose r} = {n \choose n-r}$}
${n \choose n-1} = {n \choose 1} $ difatti dato $ X $ tale che $ |X| = n $ i suoi sottoinsiemi di cardinalità 1 sono tanti quanti sono i sottoinsiemi di cardialità $n-1$. La corrispondenza biunivoca è data dall'operazione di prendere il complementare.\\
Più in generale, dato $0\leqslant r \leqslant n$, vale che $${n \choose r} = {n \choose n-r}$$
\section{Domanda 13 - ${n \choose r} = {n-1 \choose r}+{n-1 \choose r-1}$}
Dato $1\minug r\minug n-1$ $${n \choose r} = {n-1 \choose r-1}+{n-1 \choose r}$$ poichè $n\geqslant1$, posso prendere un elemento $a \in X$. Per calcolare ${n \choose r}$ devo calcolare la $|\mathcal{P}_r(X)|$. \\
Prendo $$L_1 = \{\mathcal{P}_r(X) \ | \ a \in \mathcal{P}_r(X)\}$$
$$ L_2 = \{\mathcal{P}_r(X) \ | \ a \notin \mathcal{P}_r(X)\}$$
Quindi $$\mathcal{P}_r(X) = L_1 \cup L_2 $$
Trattandosi di insiemi disgiunti: 
$${n \choose r} = |\mathcal{P}_r(X) | = |L_1| + |L_2|$$
Vediamo quindi quanto vale $|L_1|$.
$L_1$ sono tutti quei sottoinsiemi che oltre ad $a$ contengono elementi di $X$. Cioè:
$${n-1 \choose r-1} = |L_1|$$
Analogamente prendo l'insieme $X-\{a\}$, che conterrà quindi $n-1$ elementi, formando sottoinsiemi da $r$ elementi, cioè:
$${n-1 \choose r} = |L_2| $$
Quindi:
$${n \choose r} = |\mathcal{P}_r(X)| = {n-1 \choose r-1} + {n-1 \choose r}$$

\section{Domanda 17 - $\sum_{i=0}^n (-1)^i{n\choose i}$ }
Sia $n \in \campo{N}$. Quanto vale $\sum_{i=0}^n (-1)^i {n \choose i}$? Spiegare.\\
La sommatoria vale 0.

\begin{proof}
Per $n$ pari è banalmente verificato, difatti ${n \choose 1} = {n \choose n}$. Per $n$ dispari invece io la dimostro così (credo che Gaiffi ce la lasciò per esercizio):\\
Data la formula del binomio di Newton (Teorema 8.1, pag 79) 
\begin{equation}
(a+b)^n = \sum_{i=0}^n {n \choose i} a^{n-1}b^i
\end{equation}
Vogliamo ora usare questa formula per dimostrare che:
\begin{equation}
\sum_{i=0}^n (-1)^i {n \choose i} = 0
\end{equation}
Quindi scelgo opportunamente $a$ e $b$ affinché la sommatoria della formula (1) faccia 0. So che la sommatoria (1) = $(a+b)^n$. Data la somiglianza con la formula (2) pongo $a=1$ e $b=-1$, così facendo il termine a sinistra della formula (1) mi risulterà 0, mentre quello a destra risulterà:
$$\sum_{i=0}^n {n \choose i}  1^{n-1}(-1)^i$$
1 elevato a qualcosa in un prodotto non mi da alcun contributo, ergo lo posso trascurare, riordinando i termini ottengo proprio la formula (2), che è quella che volevo dimostrare. \end{proof}
\section{Domanda 20 - $ax \equiv b\ (m)$ ha soluzione se...}
Siano $a,b,c \in \campo{Z}$ con $m \geq 1$. Esporre una condizione necessaria e sufficiente perchè l'equazione $ax \equiv b \ (m) $ abbia soluzione e spiegare la motivazione.
\\ \\
L'equazione $ax \equiv b\ (m)$ non ha soluzione se $MCD(a, m)$ non divide $b$.
\begin{proof}
Se $ax \equiv b\ (m)$ ha soluzione esiste un interno $\bar{x}$ e un intero $k$ tali che $a\bar{x} = b + km$, ma supponendo che $d=MCD(a,m)$ divide $a$ e $m$ si vede subito che deve dividere anche b.
\end{proof}
L'equazione $ax \equiv b\ (m)$ ha soluzione se $MCD(a, m)$ divide $b$.
\begin{proof}
Se $MCD(a, m)$ non divide $b$ sappiamo già che la congruenza non ha soluzioni. Quindi consideriamo il caso in cui $MCD(a, m)$ divide $b$. In questo caso $MCD(a,m)$ è dunque anche il massimo fattore positivo comune a tutti e tre i numeri $a, b, m$; dividendo per $MCD(a,m)$ otteniamo la congruenza equivalente $a'x \equiv b' \  (m')$. A questo punto osserviamo che, per costruzione, $a'$ e $m'$ sono coprimi e sappiamo che in questo caso $a'$ ha un inverso $e'$ modulo $m'$. Una volta trovato $e'$ sappiamo che le soluzioni della  $a'x \equiv b' \  (m')$, sono tutti e soli gli interi della forma $e'b'+km'$ al variare di $k$ in $\campo{Z}$. Visto che $m' = \frac{m}{MCD(a, m)}$ ci sono esattamente $MCD(a,m)$ interi di questa forma in ogni sequenza di $m$ elementi consecutivi.
\end{proof}
\section{Domanda 21 - $ax+by=c$ ha soluzione se...}
\begin{lem}
L'equazione diofantea $ax+by=c$ (con a e b non entrambi nulli) ha soluzione se e solo se $MCD(a,b)$ divide c.
\end{lem}
\begin{proof}
Sappiamo, per Bezout, che se l'equazione $ax+by=c$ fosse $$ax+by=MCD(a,b)$$ questa avrebbe soluzioni certamente.\\Ma l'equazione che dobbiamo risolvere differisce da questa perchè abbiamo $c$ invece di $MCD(a,b)$. Quindi, ci basterà chiederci se $$MCD(a,b) | c\ ?$$
Se sì:\\
$\rightarrow$ L'equazione ammette soluzioni.\\
No, altrimenti.\\
Infatti si parte da una coppia $(m,n)$ che risolve l'equazione $ax+by=MCD(a,b)$: $$am+bn=MCD(a,b)$$ e si moltiplicano entrambi i membri per $k$. Troviamo allora: $$a(mk)+b(nk)=MCD(a,b)\cdot k = c$$ dunque $(mk,nk)$ è una soluzione dell'equazione iniziale.\\ \end{proof}
Viceversa, se la risposta è no, cioè $MCD(a,b)\not|\ c$, allora l'equazione non può avere soluzioni e lo possiamo dimostrare per assurdo. 
\begin{proof}
Ammettiamo che esiste una soluzione ($ \bar{x}, \bar{y}$). Consideriamo l'uguaglianza $$a\bar{x}+b\bar{y}=c$$ ricaveremo che, visto che $MCD(a,b) | a\bar{x}+b\bar{y}$ deve dividere anche quello a destra. Questo è però assurdo poichè eravamo nel caso in cui $MCD(a,b)\not|\ c$.
\end{proof}
\section{Domanda 22 - Se $ax+bt=c$ ha sol $\Rightarrow$ le sol. sono infinite}
Prendiamo l'equazione omogenea associata: $$ax+by=0$$ Cerchiamo ($\bar{x},\bar{y}$) che mi risolvono l'equazione: 
\begin{align}
ax + by &=0\\
ax &= -by\\
\frac{a}{MCD(a,b)} x &= -\frac{b}{MCD(a,b)}y
\end{align}
Questa equazioni è equivalente a quella iniziale. Supponiamo di avere una soluzione ($\gamma, \delta$): $$\frac{a}{MCD(a,b)}\gamma=-\frac{b}{MCD(a,b)}\delta$$ A questo punto, visto che i numeri $\frac{a}{MCD(a,b)}, \frac{b}{MCD(a,b)}$ sono primi fra loro, allora $\delta$ è della forma $\frac{a}{MCD(a,b)}t$ e $\gamma$ risulta uguale a $-\frac{b}{MCD(a,b)}t$. Quindi una qualunque coppia della forma $$\left(-\frac{b}{MCD(a,b)}t, \frac{a}{MCD(a,b)}t\right)$$ con $t \in \campo{Z}$ è una soluzione dell'equazione omogenea associata. 
\begin{lem} Se l'equazione diofantea ammette soluzioni, allora ammette infinite soluzioni. Presa una soluzione particolare $(\bar{x}. \bar{y})$, l'insieme S di tutte le soluzioni può essere descritto così:
$$S = \{(\bar{x}+\gamma, \bar{y} + \delta | (\gamma, \delta) \text{ è soluzione dell'equazione omogenea associata} \}$$
\end{lem}
\section{Domanda 23 - Bezout}\label{x1}
\begin{lem}[di Bezout]
Dati due interi $a$ e $b$ con $(a,b) \not= (0,0)$esistono due numeri interi $m$ e $n$ tali che $$MCD(a,b) = am+bn$$
\end{lem}
\begin{proof}
Consideriamo l'insieme $CL(a,b)$ di tutte le possibili combinazioni lineari positive a coefficienti interi di $a$ e $b$, cioè $$CL(a,b) = \{ ar+bs\ |\ r \in \campo{Z}, s \in \campo{Z}, ar+bs > 0\}$$
Tale insieme è non vuoto (difatti $(a,b) \not= (0,0)$).\\ Inoltre $CL(a,b) \subseteq \campo{N}$. Dunque per il principio del buon ordinamento ammette minimo.\\
Sia $d$ tale minimo: in particolare, dato che $d \in CL(a,b)$, esistono un $m \in \campo{Z}$ ed un $n \in \campo{Z}$ tali che $$d = am+bn$$ La dimostrazione del teorema si conclude ora mostrando che $d = MCD(a,b)$. Infatti $d$ soddisfa le proprietà del massimo comune divisore, cioè:
\begin{itemize}
\item $d\ |\ a$
\item se $c\ |\ a$ e $c\ |\ b$ allora $c\leq d$
\end{itemize}
Per il primo punto facciamo la divisione euclidea tra $a$ e $d$. Sarà $a=qd+r$ con $0\le r < d$. \\ Allora $$a=q(am+bn)+r$$ da cui $$r=(-qm+1)a+(-qn)b$$Ma allora $r$ si esprime come combinazione lineare a coefficienti interi di $a$ e di $b$. Si fosse $r>0$ avremmo che $r \in CL(a,b)$ per definizione di $CL(a,b)$. Questo non può succedere perchè $0 \le r < d$ e $d$ era stato scelto come minimo elemento di $CL(a,b)$.
Dunque deve essere $r=0$. Questo vuol dire che $a=qd+0$, ossia $d\ |\ a$. Allo stesso modo si verifica $d\ | b$. \\ \\Il secondo punto è immediato. Infatti se $c|a$ e $c|b$ allora $c\ |\ am+bn$, cioè $c|d$, in particolare $c\le d$.
\end{proof}
\section{Domanda 24 - $\cancel{d}\cdot a \equiv \cancel{d} \cdot b\ (\frac{m}{MCD(d,m)})$}
$$d\cdot a \equiv d \cdot b \ (m) \Leftrightarrow a \equiv b \ \left( \frac{m}{MCD(d,m)} \right)$$
\begin{proof}
Dimostriamo $\Leftarrow$)\\
Supponiamo che $a \equiv b \ \left( \frac{m}{MCD(d,m)} \right)$ e cerchiamo di dimostrare che $d\cdot a \equiv d \cdot b \ (m)$.\\
Da $$ a \equiv b \ \left( \frac{m}{MCD(d,m)} \right) $$ \{per definizione di congruenza\}
$$ \frac{m}{MCD(d, m)}\ |\ a-b$$
\{che equivale a dire\}
$$\frac{m}{MCD(d,m)} \cdot \gamma = a -b$$
$$m \cdot \gamma = (a-b) MCD(d,m)$$
Vorrei quindi ora dimostrare che $ m | (a-b) \cdot d \Leftrightarrow a\cdot d \equiv b \cdot d \ (m) $
$$m\cdot  \gamma \cdot d_1 = (a-b)\cdot  MCD(d,m) \cdot d_1 = (a-b) \cdot d$$
Quindi:
$$ m | (a-b)\cdot d $$ \\
Dimostriamo ora $\Rightarrow$) \\
Dal fatto che $$da \equiv db \ (m)$$ \\
\{per definizione di equivalenza\}
$$m | da - db $$
\{cioè...\}
$$m\cdot \nu = da - db = d(a-b)$$
\{divido per $MCD(d,m)$\}
$$\frac{m}{MCD(d,m)} \cdot \nu = \frac{d}{MCD(d,m)}(a-b)$$
\{Poichè $MCD \left( \frac{m}{MCD(d,m)}, \frac{d}{MCD(d,m)}\right) = 1$, cioè sono coprimi, per Bezout\}
$$MCD(d,m) = \lambda d + \mu m$$
\{divido per $MCD(d,m)$\}
$$1 = \lambda \frac{d}{MCD(d,m)} + \mu \frac{m}{MCD(d,m)}$$
Per Bezout, 1 allora è l'MCD cercato (non so cosa intendevo con  quest'ultima frase).
\\Poichè sono primi tra loro:
$$\frac{m}{MCD(d,m)} | \frac{d}{MCD(d.m)}(a-b)$$
Difatti $\frac{m}{MCD(d,m)} , \frac{d}{MCD(d.m)}$ sono primi tra loro. 
$$\frac{m}{MCD(d,m)} | (a-b)$$ cioè $$a \equiv b \ \left(\frac{m}{MCD(d,m)}\right)$$
\end{proof}
\section{Domanda 25 - Moltiplicare a destra e a sinistra una congruenza}
\begin{lem} Sia $MCD(k,m) = 1$, allora $ak \equiv bk\ (m) \Rightarrow a\equiv b\ (m)$\end{lem}
\begin{proof} 
Dall'ipotesi che $MCD(k,m) = 1$ segue che $ 1 = \lambda k + \mu m$ e quindi $\lambda$ è l'inverso di k. Moltiplicando entrambi i membri per $\lambda$ otteniamo $\lambda a k = \lambda b k\ (m)$. Siccome $\lambda k = 1 \ (m)$ otteniamo allora $$a \equiv b \ (m)$$
\end{proof}
\section{Domanda 26 - Piccolo teorema cinese del resto con moduli coprimi}
$$\bigg \{ 
\begin{array}{l}
	x \equiv a\ (m_1)\\
	x \equiv b\ (m_2)
\end{array}
$$
Osserviamo che le soluzioni della prima equazione sono $$x = a+km_1 \text{ con } k \in \campo{Z}$$
Mi chiedo se tale numero risolve la seconda soluzione. Sostituisco quindi $x$ nella seconda equazione ottendendo $$a+km_1\equiv b\ (m_2)$$
Qui la nostra variabile è quindi diventata k: $$m_1k\equiv b-a\ (m_2)$$ E sappiamo che ha soluzione solo se $MCD(m1, m2) | b-a$\\
Poichè noi abbiamo che $MCD(m_1, m_2)=1$, il nostro sistema ammetterà sempre soluzione. Tale soluzione sarà $0\le x_o < m_1m_2$. Tutte le soluzioni del sistema sono della forma $x_0 + qm_1m_2$ con $q \in \campo{Z}$.
\section{Domanda 27 - Ancora roba cinese...}
Le soluzioni di
$$ax	\equiv b\ (m_1m_2)$$
coincidono con le soluzioni di
\begin{center} 
$$\bigg \{
	\begin{array}{l}
		ax \equiv b\ (m_1) \\ 
		ax \equiv b\ (m_2)
	\end{array}
$$VERO!
\end{center}
\begin{proof}
Se $\bar{x}$ è soluzione di $$ax \equiv b\ (m_1m_2)$$ allora $\bar{x}$ è soluzione anche di $$ax \equiv b\ (m_1)$$ e di $$ax \equiv b\ (m_2)$$
Detto in altro modo $$m_1m_2 | a\bar{x}-b \Leftrightarrow m_1 | a\bar{x}-b\  \wedge \ m_2 | a\bar{x}-b$$
Viceversa se $\bar{x}$ risolve il sistema allora posso dire che $$m_1 | a\bar{x}-b \ \wedge \ m_2 | a\bar{x}-b$$ e quindi $$m_1m_2|a\bar{x}-b$$
\end{proof}
\section{Domanda 28 - $ax+by=c$ e $ax=c\ (b)$ in che modo sono collegate?}
Data $(x,y)$ la soluzione della diofantea $ax+by=c$, il numero intero $x$ deve anche soddisfare $ax\equiv c\ (b)$. Infatti $by = c-ax$, cioè $b|c-ax$. Dunque se esiste una $x$ che soddisfa $ax\equiv c\ (b)$ allora soddisfa anche $b|ax-c$ e quindi esiste una $y$ tale $by = ax-c$ trova soluzione.
\section{Domanda 29 - I numeri primi sono infiniti}
\begin{lem}
I numeri primi sono infiniti.
\end{lem}
\begin{proof}
Sia P l'insieme dei numeri primi.\\
Supponiamo per assurdo che P sia finito e dunque siano $$p_1, p_2, ..., p_n$$ tutti i numeri primi. Consideriamo allora il numero $$a = (p_1 \cdot p_2 \cdot ... \cdot p_n) + 1$$ Come accade per tutti i numeri maggiori o uguali a 2, c'è un numero primo che divide $a$. Nel nostro caso vuol dire che uno dei $p_i$ divide $a$. Ma nessuno dei nostri $p_i$ divide $a$, visto che, per ogni $i=1,2,..,n$ vale $a\equiv 1\ (mod\ p_i)$.
\end{proof}
\section{Domanda 30 - $a' = \frac{a}{MCD(a,b)}$ e $b' = \frac{b}{MCD(a,b)}$ sono coprimi}
\begin{lem}
Presi due numeri interi a e b non entrambi nulli, se li dividiamo per il loro MCD, cioè: 
$$a' = \frac{a}{MCD(a,b)} \text{ e } b' = \frac{b}{MCD(a,b)}$$  $MCD(a',b')=1$.
\end{lem}
\begin{proof}
Se ci fosse un divisore $d>1$ di $a'$ e $b'$, allora $d\cdot MCD(a,b)$ dividerebbe sia $a$ che $b$ e sarebbe più grande di $MCD(a,b)$, assurdo.
\end{proof} 
\section{Domanda 31 - Perchè l'algoritmo di Euclide funziona}
\subsection*{Perchè l'algoritmo di Euclide termina?}
\begin{align*}
a &= bq+r \\ b &= b'q'+r'\\&...
\end{align*}
$r^{n}$ nel nostro algoritmo sarà sempre $$0\le r^{n} < r^{n-1}$$ cioè prima o poi arriverà a $0$.
\subsection*{Perchè funziona?}
Funziona poichè: \begin{lem}
Se $c \equiv c'\ (m)$ allora $MCD(c,m) = MCD(c',m)$. In particolare $MCD(c,m) = MCD(Resto(c,m),m)$.
\end{lem}
\begin{proof}
Consideriamo un divisore $d$ di $m$. Allora, visto che per la definizione di congruenza deve valere $c=c'+mk$ per un certo intero $k$, possiamo concludere che $d|c \Leftrightarrow d | c'$. Quindi i divisori comuni di $m$ e $c$ coincidono con i divisori comuni di $m$ ed $c'$. Anche i massimo devono allora coincidere.
\end{proof}
\section{Domanda 32 - Criteri di divisibilità}
\subsection{Criterio di divisibilità per 3}
\subsubsection*{Cosa dice il criterio}
Sommo le cifre che compongono il numero, se il risultato che ottengo è divisibile per 3 allora anche il numero iniziale risulta divisibile per 3.
\subsubsection*{Perchè funziona}
Prendiamo ad esempio 18743291.\\
$18743291 = 1 \cdot 10^7 + 8 \cdot 10^6 + 7 \cdot 10^5 + 4 \cdot 10^4 + 3 \cdot 10^3 + 2 \cdot 10^2 + 9 \cdot 10^1 + 1$
$$10 \equiv 1 \ (mod\ 3)$$
Quindi:
$18743291 = 1+8+7+4+3+2+9+1 = 8 = 2 \ (mod\ 3)$
\subsection{Criterio di divisibilità per 7}
Cerco un multiplo di 10 "comodo" per l'operazione $mod\ 7$
\begin{align*}
10 &\equiv 3\ (7)\\
100 &\equiv 2\ (7) \\
1000 &\equiv -1\ (7)
\end{align*}
Prendiamo 3417822.\\
$$3417822 = 3 \cdot 1000^2 + 417 \cdot 1000 + 822$$
Essendo $1000 \equiv -1\ (7)$: 
\begin{align*}
3417822 &= 3 \cdot (-1)^2 + 417 \cdot (-1) + 822\\
&= 3 -417 +822\\ &= 408 \\ &= 7\cdot 58 + 2\\ &\equiv 2\ (mod\ 7)
\end{align*}
\subsection{Criterio di divisibilità per 11}
$$10\equiv -1 \ (11)$$
Prendiamo 78922. \\
\begin{align*}
78922 &= 7\cdot 10^4+ 8\cdot 10^3 + 9 \cdot 10^2 + 2 \cdot 10 + 2\\
&\equiv 7 \cdot (-1)^4 + 8 \cdot (-1)^4 + 9 \cdot (-1)^2 + 2 \cdot (-1) + 2\\
&=7-8+9-2+2\\
&\equiv 8\ (mod\ 11)
\end{align*}

\section{Domanda 33 - Classi di resto et alia}
Sia $m$ un numero intero positivo. Per ogni $i = 0,1,2,...,m-1$ chiamiamo $[i]_m$ la "classe di resto di $i$ modulo $m$", ossia l'insieme dei numeri che danno resto $i$ quando si considerano la loro divisione euclidea per $m$: $$[i]_m = \{x \in \campo{Z}\ |\ x \equiv i\ (m)\}$$
Definiamo poi $\campo{Z}_m$ l'insieme di tutte le classi di resto modulo $m$: $$\campo{Z}_m = \{ [0]_m, [1]_m, ..., [m-1]_m \}$$
\begin{lem}
Se $p$ è un numero primo, allora $\campo{Z}_p$ è un campo.
\end{lem}
\begin{proof}
Se prendiamo una classe di resto $[a]_p \not= [0]_p$ in $\campo{Z}_p$, allora deve valere che $MCD(a, p) = 1$. allora la congruenza $ax\equiv 1\ (p)$ ha soluzione, dunque esiste $b \in \campo{Z}$ tale che $ab \equiv 1\ (p)$. Come conseguenza in $\campo{Z}_p$ vale $$[a]_p[b]_p = [ab]_p = [1]_p$$
Abbiamo allora dimostrato che $[a]_p$ è invertibile in $\campo{Z}_p$ e che $[b]_p$ è il suo inverso. Quindi $\campo{Z}_p$ ha l'inverso per ogni numero $\Rightarrow$ è un campo.
\end{proof}
\section{Domanda 34 - Piccolo Teorema di Fermat}
\begin{lem}[Il piccolo teorema di Fermat]
Se p è un numero primo e a è un numero intero che non è multiplo di p, allora vale che $$a^{p-1} \equiv 1\ (p)$$
\end{lem}
\begin{proof}
Consideriamo l'anello $\campo{Z}_p$ delle classi di resto modulo $p$: $$\campo{Z}_p = \{[0], [1], ..., [p-1]\}$$ Vista la scelta di $a$, sappiamo che $[a] \not= [0]$. Moltiplichiamo ora tutti gli elementi di $\campo{Z}_p$ per $[a]$: $$[a][0], [a][1], ..., [a][p-1]$$ Questi $p$ elementi sono tutti diversi fra loro? \\
Se sì $\Rightarrow$ sappiamo esattamente tutti gli elementi di $\campo{Z}_p$, che ha cardinalità $p$.\\
Verifichiamo dunque che sono tutti diversi fra loro: supponiamo, per assurdo, che esistano $i$ e $j$, con ($0 \le i \le j \le p-1$) con $[i] \not=[j]$ ma tali che $[a][i] = [a][j]$.\\ \\ Poichè $p$ è primo, $\campo{Z}_p$ è un campo, quindi ogni elemento $\in \campo{Z}_p$ ha un inverso. Sia dunque $[b]$ l'inverso di $[a]$. Moltiplicando per $[b]$ otteniamo: $$[b][a][i] = [b][a][j]$$ Siccome $[a][b]=1$ $$[i] = [j]$$Poichè avevamo supposto $[i] \not= [j]$, abbiamo trovato un assurdo.\\
\\
Visto ora che sono tutti distinti, sappiamo che la lista $$[a][0], [a][1], ..., [a][p-1]$$ ha esattamente tutti gli elementi di $\campo{Z}_p$. allora facciamo il prodotto degli elementi di questa lista, eccetto di $[a][0] = [0]$, deve valere $$[a][1] ...  [a][p-1] = [1][2][3]...[p-1]$$ visto che nel membro a sinistra e in quello a destra abbiamo tutti gli elementi (magari in ordine diverso). \\Per la proprietà commutativa possiamo riscrivere l'uguaglianza nella forma $$[a]^{p-1}[1] ...[p-2][p-1] = [1][2][3]...[p-2][p-1]$$ Poichè $[p-1]$ è invertibile in $\campo{Z}_p$, moltiplichiamo entrambi i membri per il suo inverso. Otteniamo quindi $$[a]^{p-1}[1] ... [p-2] = [1][2][3] ... [p-2]$$ Poi moltiplichiamo entrambi i membri per l'inverso di $[p-2]$, poi di $[p-3]$ e così via...\\
Alla fine troviamo $$[a]^{p-1} = [1]$$ che si riscrive, in termini di congruenze come $$a^{p-1} \equiv 1\ (p)$$ che è proprio l'enunciato che volevamo dimostrare.
\end{proof}
\section{Domanda 36 - $a^{561} \equiv a\ (561)$}
Vale poichè $561$ è un numero di Carmichael. 
\href{http://it.wikipedia.org/wiki/Numero_di_Carmichael}{Approfondisci su Wikipedia $>>$}
\section{Domanda 44 - Scrittura unica per i vettori}
\begin{lem}
Ogni elemento di uno spazio vettoriale si scrive in modo unico come combinazione lineare degli elementi di una base.
\end{lem}
\begin{proof}
Prendiamo la base B di V $$V = \{v_1, v_2, ..., v_n\}$$
Prendiamo il vettore $$q = \alpha_1 v_1 + \alpha_2 v_2 + \alpha_3 v_3 +...+ \alpha_n v_n$$ e prendiamo anche il vettore $$t = \beta_1v_1+\beta_2v_2+\beta_3v_3+...+ \beta_nv_n$$. Visto che questi due vettori sono uguali 
$$q= t$$
$$q-t = 0$$
$$\alpha_1 v_1 + \alpha_2 v_2 + \alpha_3 v_3 +...+ \alpha_n v_n - ( \beta_1v_1+\beta_2v_2+\beta_3v_3+...+ \beta_nv_n ) = 0$$
$$(\alpha_1-\beta_1)v_1+(\alpha_2-\beta_2)v_2+...+(\alpha_n-\beta_n)v_3=0$$
Poichè $v_i \not= 0 \forall i$ devono essere $=0$ i coefficienti $\alpha_i\beta_i \forall i$. Quindi: $$\alpha_i-\beta_i=0$$ $$\alpha_i=\beta_i$$
\end{proof}
\section{Domanda 45 - Scarti successivi per estrarre base}
\begin{lem}
	Sia V uno spazio vettoriale di dimensione finita su $\campo{K}$. Da ogni insieme di generatori di V si può estrarre una base. Formalmente: $$\forall g \subseteq V <g> = V \Rightarrow \exists B \subseteq g | B \text{ è una base di V})$$
\end{lem}
\begin{proof}
	Se g è linearmente indipendente allora questo è già una base di V.\\
	Se invece g è composto da elementi linearmente dipendenti, ovvero $\alpha_1v_1+...+\alpha_nv_n=0$ per $a_i$ non tutti nulli $\Rightarrow \exists i | \alpha_i \not= 0$ (esiste un elemento non nullo) \\ ma allora $$v_i = \frac{-1}{\alpha_i}(\alpha_1v_1+...+\cancel{a_iv_i}+...+\alpha_nv_n)$$ e quindi abbiamo scoperto che $v_i \in <v_1, ..., v_n>$ cioè $v_i$ è combinazione lineare di $ g - \{v_i\}$. \\Questo vuol dire che l'insieme $g$ si può ridurre eliminando l'elemento a esso dipendente.
\end{proof}
\section{Domanda 48 - Numero di pivot non dipende dalla riduzione a scala}
Saper spiegare perchè il numero di pivot di una matrice non dipende dalla riduzione a scala effettuata. Definizione di rango di una matrice.\\ \\
Diciamo innanzitutto che il rango della matrice è il numero di pivot che presenta la matrice. I pivot sono il numero di righe in una matrice a scala che presentano come primo elemento un numero non nullo.\\
Ci basta ora dimostrare quindi che la riduzione a scala effettuata, quindi le operazioni elementari di riga effettuate, non alterano il rango di una matrice.
\begin{lem}
Siano $\{v_1, ..., v_m\} \in \campo{K}^n\ | <v_1, ..., v_m> = V \subseteq \campo{K}^n$.\\
Sia A una matrice e S la sia S la sua riduzione a scala.\\
Siano $j_1, ..., j_r$ le colonne che contengono i pivot. Allora i vettori $V_{j_1},..., v_{j_r}$ (della matrice A) sono una base di V estratta dall'insieme dei generatori   $\{v_1, ..., v_m\}$
\end{lem}
\begin{proof}
Dimostriamo che $V_{j_1},..., v_{j_r}$ sono una base di V.
\\1) Sono linearmente indipendenti.
Data la matrice $M$, le colonne di $M$ sono linearmente indipendenti $\Leftrightarrow Mx = 0$ ha come unica soluzione $x = 0$ cioè, $\Leftrightarrow rkM = |\text{Colonne di }M|$\\ \\
Calcoliamo quindi il rango di $M$ e verificare che sia $r$.\\
Agisco su $M$ con le stesse operazioni fatte su $A$ per ottenere $S$. Il numero di Pivot che ottengo è ancora r e quindi $rkM=r$. ciò dimostra che $\{V_{j_1},..., v_{j_r}\}$ sono linearmente indipendenti.\\ \\
2) Generano V, cioè che $<V_{j_1},..., v_{j_r}> = <V_{1},..., v_{m}> = V$\\
Vediamo se $<V_{j_1},..., v_{j_r}>\subseteq <V_{1},..., v_{m}>$. \\Quest'inclusione è ovvia (mmm, mica tanto ovvia!)\\Vediamo ora l'inclusione opposta, cioè:
$\forall i\ v_i \in <V_{j_1},..., v_{j_r}> $, cioè che ogni vettore è generato da $<V_{j_1},..., v_{j_r}>$. \\
Perchè questo sia vero occorre che $Mx=b$ abbia soluzione. $Mx=b$ ha soluzione se $b \in \text{colonne di M } (V_{j_1},..., v_{j_r})$\\
Applichiamo quindi le stesso operazioni fatte su $A$ per ottenere $S$ sulla matrice $M|b$ con $b=v_i$.
$$(M|v_i) \xrightarrow{Gauss} S_0 = (V_{j_1},..., v_{j_r} | v_i)$$ e si nota facilmente che $rkM = rkS_o = r$ quindi genera.
\end{proof}


\section{Domanda 50 - Inettività $\Leftrightarrow$ $ker = \{O\}$}
\begin{lem}
L'applicazione lineare $f: X \rightarrow Y$ è iniettiva $\Leftrightarrow$ $Kerf = \{O\}$
\end{lem}
\begin{proof} Dimostro $\Rightarrow$)\\
Supponiamo $f$ iniettiva. Se $x \in Ker f$ $$f(x) = 0 = f(0) \Rightarrow x = 0$$ dunque $Ker f = 0$.\\
($f(0) =  0$ poichè l'applicazione è \textbf{lineare}, poichè avevamo supposto che fosse iniettiva possiamo dire che quella $x$ che abbiamo preso è l'unica $x$ che va in $0$.)\\ \\
Dimostro $\Leftarrow$)\\
Supponiamo che $Ker f = 0$ 
\begin{align*}
f(x) &= f(y)\\
 f(x-y)&=0\\ \Rightarrow x-y &\in Kerf \\
x-y &= 0\\ x &= y
\end{align*}
Dunque, essendo $x=y$ segue che $f$ è iniettiva.
\end{proof}
\section{Domanda 53 - Teorema della dimensione}
\begin{lem}[della dimensione]Sia F un'applicazione lineare da $X \rightarrow Y$. \\ Dim Dom F= Dim Im F+ Dim Ker F\end{lem}
\begin{proof}
Sia $\{u_1, ... , u_r\}$ una base di $KerF$. La completo a base di X: $\{u_1, ... , u_r, v_{r+1}, ..., u_n\}$ \{se $KerF = \{O\}$, prendiamo direttamente una base $\{v_1, ..., v_n\}$ di $V$, e consideriamo $r=0$ ed $s=n$ nel seguito\}. Poniamo $w_j = F(v_{r+j}) \in W$ per $j=1, ..., s = n-r$; se dimostraimo che $F(u_i)=O$ per $i=1,...,r$) sappiamo già che $B$ è un sistema di generatori di $ImF$; dobbiamo solo far vedere che $w_1, ..., w_s \in \campo{R}$ siano tali che $$\alpha_1 w_1 + ... + \alpha_s w_s = O$$
Allora
$$O = \alpha_1F(v_{r+1})+...+\alpha_sF(v_{r+s}) = F(\alpha_1v_{r+1}+...+\alpha_sv_{r+s})$$
per cui $\alpha_1v_{r+1}+...+\alpha_sv_{r+s} \in KerT$. Questo vuol dire che esistono $\beta_1, ..., \beta_r \in \campo{R}$ tali che $\alpha_1v_{r+1}+...+\alpha_sv_{r+s} = \beta_1u_1+...+\beta_ru_r$; quindi $$\beta_1u_1+...+\beta_ru_r-a_1v_{r+1}-...-\alpha_sv_{r+s}=O$$
e l'indipendenza lineare di $\{u_1, ..., u_r, v_{r+1},...,v_{r+s}\}$ implica $\alpha_1=...=\alpha_s=0$, come desiderato.
\end{proof}

\section{Domanda 58 - Matrice associata e cambiamento di base}
\subsection*{Matrice associata ad un'applicazione lineare}
La matrice associata ad un applicazione lineare è una matrice che ha per colonne tutti i vettori della base dell'immagine dell'applicazione lineare.
Formalmente:
\begin{definizione}
Sia $f:X\rightarrow Y$ un'applicazione lineare. Sia $\mathscr{B}=\{v_1, ...,v_n\}$ una base di X. \\La matrice $A$ associata all'applicazione lineare sarà così formata:\\
$$A = \left ( 
	\begin{array}{c c c c}
		\left(\begin{array}{c}
			| \\
			fv_1\\
			|
			\end{array}
		\right) &
		\left(\begin{array}{c}
			| \\
			fv_2\\
			|
			\end{array}
		\right) & ... & \left(\begin{array}{c}
			| \\
			fv_n\\
			|
			\end{array}
		\right) 
	\end{array}
\right)
$$
\end{definizione}
\subsection*{Cambiamento di base}
Partiamo da un esempio esplicativo per capire il cambiamento di base: \\
Sia $$B = \left\{ \left( \begin{array}{c} 1\\1\\1\end{array} \right), \left( \begin{array}{c} 2\\0\\1\end{array} \right),\left( \begin{array}{c} 0\\1\\5\end{array} \right)\right\}$$
e sia
$$C = \left\{ \left( \begin{array}{c} 1\\0\\0\end{array} \right), \left( \begin{array}{c} 0\\1\\0\end{array} \right),\left( \begin{array}{c} 0\\0\\1\end{array} \right)\right\}$$
Se volessi scrivere il vettore $\left( \begin{array}{c} 3\\1\\2\end{array} \right)$ come combinazione della base B dovrei scriverlo così: $$\left( \begin{array}{c} 1\\1\\0\end{array} \right)_B$$
mentre lo stesso vettore scritto rispetto alla base C lo scrivo così: $$\left( \begin{array}{c} 3\\1\\2\end{array} \right)_C$$
Ora quindi ci serve solamente trovare un modo per fare questi passaggi più velocemente possibile. Poichè un'applicazione lineare è definita su una base, se io cambio la base cambio l'intera applicazione. Prendiamo la nostra matrice associata all'applicazione lineare $$A = \left ( 
	\begin{array}{c c c c}
		\left(\begin{array}{c}
			| \\
			fv_1\\
			|
			\end{array}
		\right) &
		\left(\begin{array}{c}
			| \\
			fv_2\\
			|
			\end{array}
		\right) & ... & \left(\begin{array}{c}
			| \\
			fv_n\\
			|
			\end{array}
		\right) 
	\end{array}
\right)
$$
Per rendere A una matrice del cambiamento di base devo solamente riscrivere i suoi vettori nella nuova base, cioè:
$$A = \left ( 
	\begin{array}{c c c c}
		\left(\begin{array}{c}
			| \\
			fv_1\\
			|
			\end{array}
		\right)_{B_2} &
		\left(\begin{array}{c}
			| \\
			fv_2\\
			|
			\end{array}
		\right)_{B_2} & ... & \left(\begin{array}{c}
			| \\
			fv_n\\
			|
			\end{array}
		\right)_{B_2} 
	\end{array}
\right)
$$
Quindi avrò così ottenuto una matrice $A'$ con i miei vettori della base $B_1$ in partenza e quelli della base $B_2$ in arrivo. \\La matrice quindi cambia come cambiano i vettori in essa contenuti rispetto alla nuova base, cioè, al variare della base d'arrivo varieranno i vettori all'interno della matrice $A$.

\section{$L_A$ è invertibile $\Leftrightarrow$ $A$ non è singolare}
\begin{definizione}
Una matrice $A$ si dice singolare se $rk A = max $
\end{definizione}
\begin{lem}
Data un'applicazione lineare $L_A$, quest'applicazione è invertibile $\Leftrightarrow$ la matrice $A$ associata ad $L_A$ è non singolare.
\end{lem}
\begin{proof}
$L_A$ è invertibile $\Leftrightarrow$ è iniettiva e surgettiva $\Leftrightarrow L_A$ è surgettiva.
$$\Leftrightarrow ImL_A = \campo{K}^n$$ e sono uguali $$\Leftrightarrow DimL_A = Dim\campo{K}^n = n \Rightarrow rkA=n $$$\Rightarrow A$ non è singolare.
\end{proof}
\end{document}
